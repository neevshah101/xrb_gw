%% Beginning of file 'sample701.tex'
%%
%% Version 7.0.1. Created May 2025.
%% Version 7. Created January 2025.
%%
%% AASTeX v7+ calls the following external packages:
%% times, hyperref, ifthen, hyphens, longtable, xcolor,
%% bookmarks, array, rotating, ulem, and lineno
%%
%% RevTeX is no longer used in AASTeX v7+.
%%
\documentclass[linenumbers,trackchanges,twocolumn]{aastex701}
%%
%% This initial command takes arguments that can be used to easily modify
%% the output of the compiled manuscript. Any combination of arguments can be
%% invoked like this:
%%
%% \documentclass[argument1,argument2,argument3,...]{aastex701}
%%
%% Six of the arguments are typestting options. They are:
%%
%%  twocolumn   : two text columns, 10 point font, single spaced article.
%%                This is the most compact and represent the final published
%%                derived PDF copy of the accepted manuscript from the publisher
%%  default     : one text column, 10 point font, single spaced (default).
%%  manuscript  : one text column, 12 point font, double spaced article.
%%  preprint    : one text column, 12 point font, single spaced article.
%%  preprint2   : two text columns, 12 point font, single spaced article.
%%  modern      : a stylish, single text column, 12 point font, article with
%% 		  wider left and right margins. This uses the Daniel
%% 		  Foreman-Mackey and David Hogg design.
%%
%% Note that you can submit to the AAS Journals in any of these 6 styles.
%%
%% There are other optional arguments one can invoke to allow other stylistic
%% actions. The available options are:
%%
%%   astrosymb    : Loads Astrosymb font and define \astrocommands.
%%   tighten      : Makes baselineskip slightly smaller, only works with
%%                  the twocolumn substyle.
%%   times        : uses times font instead of the default.
%%   linenumbers  : turn on linenumbering. Note this is mandatory for AAS
%%                  Journal submissions and revisions.
%%   trackchanges : Shows added text in bold.
%%   longauthor   : Do not use the more compressed footnote style (default) for
%%                  the author/collaboration/affiliations. Instead print all
%%                  affiliation information after each name. Creates a much
%%                  longer author list but may be desirable for short
%%                  author papers.
%% twocolappendix : make 2 column appendix.
%%   anonymous    : Do not show the authors, affiliations, acknowledgments,
%%                  and author contributions for dual anonymous review.
%%  resetfootnote : Reset footnotes to 1 in the body of the manuscript.
%%                  Useful when there are a lot of authors and affiliations
%%		    in the front matter.
%%   longbib      : Print article titles in the references. This option
%% 		    is mandatory for PSJ manuscripts.
%%
%% Since v6, AASTeX has included \hyperref support. While we have built in
%% specific %% defaults into the classfile you can manually override them
%% with the \hypersetup command. For example,
%%
%% \hypersetup{linkcolor=red,citecolor=green,filecolor=cyan,urlcolor=magenta}
%%
%% will change the color of the internal links to red, the links to the
%% bibliography to green, the file links to cyan, and the external links to
%% magenta. Additional information on \hyperref options can be found here:
%% https://www.tug.org/applications/hyperref/manual.html#x1-40003
%%
%% The "bookmarks" has been changed to "true" in hyperref
%% to improve the accessibility of the compiled pdf file.
%%
%% If you want to create your own macros, you can do so
%% using \newcommand. Your macros should appear before
%% the \begin{document} command.
%%
\newcommand{\vdag}{(v)^\dagger}
\newcommand\aastex{AAS\TeX}
\newcommand\latex{La\TeX}

\newcommand{\Mdot}{\mathrm{M}_{\odot}}
\newcommand{\Rdot}{\mathrm{R}_{\odot}}
\newcommand{\Ldot}{\mathrm{L}_{\odot}}
\newcommand{\Myr}{\mathrm{Myr}}
\newcommand{\red}{\textcolor{red}}
\newcommand{\mr}[1]{\textcolor{green!70!black}{#1}}

%%%%%%%%%%%%%%%%%%%%%%%%%%%%%%%%%%%%%%%%%%%%%%%%%%%%%%%%%%%%%%%%%%%%%%%%%%%%%%%%
%%
%% The following section outlines numerous optional output that
%% can be displayed in the front matter or as running meta-data.
%%
%% Running header information. A short title on odd pages and
%% short author list on even pages. Note that this
%% information may be modified in production.
%%\shorttitle{AASTeX v7.0.1 Sample article}
%%\shortauthors{The Terra Mater collaboration}
%%
%% Include dates for submitted, revised, and accepted.
%%\received{February 1, 2025}
%%\revised{March 1, 2025}
%%\accepted{\today}
%%
%% Indicate AAS Journal the manuscript was submitted to.
%%\submitjournal{PSJ}
%% Note that this command adds "Submitted to " the argument.
%%
%% You can add a light gray and diagonal water-mark to the first page
%% with this command:
%% \watermark{text}
%% where "text", e.g. DRAFT, is the text to appear.  If the text is
%% long you can control the water-mark size with:
%% \setwatermarkfontsize{dimension}
%% where dimension is any recognized LaTeX dimension, e.g. pt, in, etc.
%%%%%%%%%%%%%%%%%%%%%%%%%%%%%%%%%%%%%%%%%%%%%%%%%%%%%%%%%%%%%%%%%%%%%%%%%%%%%%%%
%%
%% Use this command to indicate a subdirectory where figures are located.
%%\graphicspath{{./}{figures/}}
%% This is the end of the preamble.  Indicate the beginning of the
%% manuscript itself with \begin{document}.

\begin{document}

\title{A Tale of Two Siblings: GW190814 and a Galactic High-Mass X-ray Binary}%\footnote{Footnotes can be added to titles}}

%% A significant change from AASTeX v6+ is in the author blocks. Now an email
%% address is required for each author. This means that each author requires
%% at least one of the following:
%%
%% \author
%% \affiliation
%% \email
%%
%% If these three commands are not available for each author, the latex
%% compiler will issue an error and if you force the latex compiler to continue,
%% it will generate an incomplete pdf.
%%
%% Multiple \affiliation commands are allowed and authors can also include
%% an optional \altaffiliation to indicate a status, i.e. Hubble Fellow.
%% while affiliations are indexed as footnotes, altaffiliations are noted with
%% with a non-numeric footnote that is set away from the numeric \affiliation
%% footnotes. NOTE that if an \altaffiliation command is used it must
%% come BEFORE the \affiliation call, right after the \author command, in
%% order to place the footnotes in the proper location. Because non-numeric
%% symbols are used, \altaffiliation should be used sparingly.
%%
%% In v7+ the \author command takes an optional argument which provides
%% additional metadata about the author. Authors can provide the 16 digit
%% ORCID, the surname (family or last) name, the given (first or fore-) name,
%% and a name suffix, e.g. "Jr.". The syntax is:
%%
%% \author[orcid=0000-0002-9072-1121,gname=Gregory,sname=Schwarz]{Greg Schwarz}
%%
%% This name metadata in not shown, it is only for parsing by the peer review
%% system so authors can be more easily identified. This name information will
%% also be sent to the publisher so they can include it in the CROSSREF
%% metadata. Including an orcid will hyperlink the author name to the
%% author's ORCID page. Note that  during compilation, LaTeX will do some
%% limited checking of the format of the ID to make sure it is valid. If
%% the "orcid-ID.png" image file is  present or in the LaTeX pathway, the
%% ORCID icon will appear next to the authors name.
%%
%% Even though emails are now required for each author, the \email does not
%% produce output in the compiled manuscript unless the optional "show" command
%% is used. For example,
%%
%% \email[show]{greg.schwarz@aas.org}
%%
%% All "shown" emails are show in the bottom left of the first page. Due to
%% space constraints, only a few emails should be shown.
%%
%% To identify a corresponding author, use the \correspondingauthor command.
%% The command appends "Corresponding Author: " to the argument it appears at
%% the bottom left of the first page like the output from \email.

\author[orcid=0000-0002-8465-8090,sname='North America']{Neev Shah}
\affiliation{Department of Astrnomy \& Steward Observatory, The University of Arizona}
\email[show]{neevshah@arizona.edu}

% \author[orcid=0000-0000-0000-0002,gname=Bosque, sname='Sur America']{Mathieu Renzo}
% \affiliation{Steward Observatory, The University of Arizona}
% \email{mrenzo@arizona.edu}

% \author[orcid=0000-0000-0000-0002,gname=Bosque, sname='Sur America']{Koushik Sen}
% \affiliation{Steward Observatory, The University of Arizona}
% \email{ksen@arizona.edu}

% \author[orcid=0000-0000-0000-0002,gname=Bosque, sname='Sur America']{Aldana Grichener}
% \affiliation{Steward Observatory, The University of Arizona}
% \email{agrichener@arizona.edu}

%% Use the \collaboration command to identify collaborations. This command
%% takes an optional argument that is either a number or the word "all"
%% which tells the compiler how many of the authors above the command to
%% show. For example "\collaboration[all]{(DELVE Collaboration)}" wil include
%% all the authors above this command.
%%
%% Mark off the abstract in the ``abstract'' environment.
\begin{abstract}

\red{TBA}

\end{abstract}

%% Keywords should appear after the \end{abstract} command.
%% The AAS Journals now uses Unified Astronomy Thesaurus (UAT) concepts:
%% https://astrothesaurus.org
%% You will be asked to selected these concepts during the submission process
%% but this old "keyword" functionality is maintained in case authors want
%% to include these concepts in their preprints.
%%
%% You can use the \uat command to link your UAT concepts back its source.
\keywords{\red{TBA}}%\uat{Galaxies}{573} --- \uat{Cosmology}{343} --- \uat{High Energy astrophysics}{739} --- \uat{Interstellar medium}{847} --- \uat{Stellar astronomy}{1583} --- \uat{Solar physics}{1476}}

\section{Introduction}

Massive stars ($> 8-10\,\Mdot$) end their lives in energetic explosions or implosions, leaving behind a compact object remnant like a black hole (BH) or a neutron star (NS) \citep{2025arXiv250214836J}. Since most massive stars live in binaries and higher order systems \citep{2012Sci...337..444S,2017ApJS..230...15M, 2023ASPC..534..275O}, they are likely to interact at some point in their short lifetimes \red{(few $\mathrm{Myr}$)}. Such systems are thought to be the progenitors of a wide range of high energy phenomena, such as X-ray binaries (XRB) \citep{2006csxs.book..623T}, stripped envelope supernovae \citep{2011MNRAS.412.1522S,2016MNRAS.457..328L}, gravitational wave (GW) sources \citep{2017ApJ...846..170T, 2022PhR...955....1M, 2023PhRvX..13d1039A}, gamma--ray bursts (GRB) \citep{2007A&A...465L..29C}, and various other transients \citep{2012ApJ...752L...2C,2022ApJ...932...84M,2023MNRAS.523.6041G}.

A particularly interesting outcome of massive binary evolution is the presence of compact objects in binaries. In the local universe, these systems have been observed for decades as X-ray binaries. XRBs are an important probe to study the properties of compact objects and the physical processes occuring in the accretion disks around them. They are also powerful sources of ionizing radiation that can be important for stellar feedback on galactic scales \citep{2012MNRAS.423.1641J}. More recently, astrometry with Gaia, and other radial velocity surveys have unraveled a small but growing population of non--interacting (detached) binaries that contain a compact object such as a BH \citep{2023MNRAS.518.1057E,2023MNRAS.521.4323E,2024A&A...686L...2G} or NS \citep{2024OJAp....7E..58E}. An even rarer outcome of massive binary evolution are double compact binaries. Although a few binary neutron star (BNS) systems have been observed in the Galaxy through pulsar observations, double compact binaries are also strong sources of GWs, especially when they are close to merger. GWs from merging binary black holes (BBH) were directly detected for the first time in 2015. The sample has been rapidly growing since, with more than $200$ mergers in the latest catalog, GWTC-4.0 \citep{2025arXiv250818082T,2025arXiv250818083T} which consists mostly of BBH's and a handful of BNS and NSBH coalescences. In the catalog are a few exceptional events, such as the most asymmetric mass-ratio merger, GW190814, between a compact object of $\sim 2.6\,\Mdot$, and a $\sim 23\,\Mdot$ BH \citep{2020ApJ...896L..44A}. The highly asymmetric mass ratio of this event, and the presence of a compact object in the lower--mass-gap has been challenging to interpret for most formation channels. Various attempts have been made to potentially explain its formation, such as through isolated binary evolution \citep{2020ApJ...899L...1Z,2025arXiv251116648M}, a triple scenario where the $2.6\,\Mdot$ compact object is the remnant of a previous BNS merger \citep{2021MNRAS.500.1817L}, and through dynamical interactions in young star clusters \citep{2021ApJ...908L..38A}. Due to the unknown BH or NS nature of the lower mass compact object, it is also often categorized as an outlier to the BBH population. Outlier events challenge conventional channels for the formation of such systems. Due to their very nature of not fitting in well with the rest of the population, including or excluding them from analyses can significantly bias the inferred population parameters \citep{2025arXiv250814159C,2021ApJ...913L...7A,2022ApJ...926...34E,2022ApJ...931..108F}. It is important to carefully characterize and study them so as to understand whether they are true outliers that hint towards an exotic formation scenario, or are they just rare outcomes of an otherwise normal evolutionary pathway.

As GW events are a particularly rare outcome of massive binary evolution, it is often challenging to reconstruct their past evolutionary history on an individual  event--by--event basis. As a result, most studies search for signatures of different formation channels in the overall population properties, such as the mass, redshift and spin distributions, and possible correlations among them. However, another way of understanding the formation of GW sources is by finding and characterizing their analogs in intermediate evolutionary stages. Such systems can be observed and well studied in the Milky Way and nearby galaxies. These can potentially be important anchors in understanding the pathway to forming a potential GW source, and provide constraints on uncertainties in massive binary evolution, such as the details of the mass transfer process, the explodability and natal kick received to compact remnants, and uncertainties in common envelope evolution. These uncertainties propagate in forming a GW source, which makes their astrophysical interpretation a challenging task. Utilizing local analogs can be particularly useful for understanding the formation of outlier events such as GW190814, which do not necessarily fit in well wit the rest of the GW population. A key intermediate stage in the pathway to becoming a gravitational wave source is a potential XRB phase. Although few presently observed XRBs in the local universe will result in merging compact binaries in the future \citep{2006ARA&A..44...49R, 2022ApJ...929L..26F}, any compact binary merger originating from isolated binary evolution will experience a (brief) XRB during its evolution.

Given the extreme mass--ratio of GW190814, a potential local analog for it is the HMXB 4U 1700-37/ HD 153919 \citep{2015A&A...577A.130F}. One of the first X--ray sources discovered in the 1970s, it consists of an O6.5 Iaf+ star, which is one of the most massive stars in known XRBs with an estimated mass of $40-60\,\Mdot$. The compact object in this system is equally interesting, with initial estimates putting its mass at $\approx 2.5\,\Mdot$, straddling the ``mass--gap" between the heaviest neutron stars and lightest black holes. However, recent observations hint that its mass may be lower at $\approx 2\,\Mdot$, and that it is more likely to be a NS. With the help of Gaia EDR3, and verified by \citep{2022MNRAS.511.4123H} using Gaia DR3, \citep{2021A&A...655A..31V} confirmed that this system originated from the young open cluster NGC 6231, and was ejected from it $\approx 2\,\Myr$ ago. Together with the young age of the NGC 6231 (few $\Myr$), this implies that the star that exploded had an initial mass $>\,30\,\Mdot$. The short period of the binary ($3.41$ days), highly asymmetric mass ratio, and age constraints have been challenging to interpret given our standard understand of binary evolution. \citep{2021A&A...655A..31V} proposed the system to have formed from a pair of massive stars that underwent mass transfer (MT) while the more massive (donor) star was still on the Main Sequence (MS), termed as Case A MT. Given the short period of the binary today, they also hypothesized that the compact remnant received a natal kick in a favorable direction that shrinks the orbit. The presence of a compact object in the mass--gap, and the highly asymmetric mass ratio of this system makes it an interesting case study for understanding the formation of GW190814, as they could potentially share a common evolutionary pathway.

Our goal in this work is two--fold. We first utilize the rich set of observational constraints for the HMXB 4U 1700-37/ HD 153919 (henceforth XRB) to estimate the properties of the binaries just prior to the first (and only) SN in the system. With the help of these constraints, we construct detailed binary evolution models computed with MESA to chart its past evolutionary history and likely progenitor system. In doing so, we demonstrate that forming this XRB through Case A MT is viable, as proposed originally in \citep{2021A&A...655A..31V}. We also show that a similar formation pathway can also work at lower metallicity, and lead to compact binaries with masses similar to what is seen in GW190814. We emphasize that our progenitor systems for explaining both the XRB and GW190814 are \textit{``fairly normal"}, and the conservativeness of mass transfer, along with the strength and direction of natal kicks received at the time of first SN are crucial in determining the final fate and rates of such systems.

In Section \ref{sec:monte-carlo}, we describe our monte carlo simulations, and use it to reconstruct the pre-SN properties of the HMXB 4U 1700-37. Motivated by those results, in Section \ref{xrb} we use \textsc{MESA} to construct an evolutionary pathway for this system starting from a massive binary. We also discuss the potential future evolution of the HMXB. In Section \ref{sec:gw190814}, we explore the formation of GW190814-like systems. We discuss the conditions in which such a system can form, and its similarities and differences with the HMXB described earlier. We also make back--off--the--envelope estimates for the rates of such events in Section \ref{rates}. In Section (Discussion and Conclusion), we summarize our results and discuss the implications of our work.

\section{Reconstruction of the HMXB 4U 1700-37 pre-supernova} \label{sec:monte-carlo}

\subsection{Methods} \label{sec:monte-carlo-method}

The occurence of a SN in a binary can have a significant effect on the orbital properties of the binary \red{\citep{1994astro.ph.12023B,2025OJAp....8E..85W}}. There can be diverse post--SN outcomes in a binary depending on the amount of mass lost in the explosion. Due to asymmetries in the explosion and/or neutrino emission, the compact remnants can also receive a natal kick (\cite{1975Natur.253..698K}, see \cite{2025NewAR.10101734P} for a review). The strength and directions of natal kicks, and its dependence on the nature of the compact remnant (NS or BH) are not well known (\cite{2024PhRvL.132s1403V, 2025PASP..137c4203N,2025ApJ...989L...8D, 2025arXiv250508857V}), and can have a large impact on the post--SN binary orbit. This can lead to several observational consequences such as binary disruption, and if it remains bound, a change in the orbital period, eccentricity and even a systemic velocity to the binary as a whole. These changes do not require the compact object to receive a natal kick, as just the momentum carried by the lost ejecta in a SN explosion can change the binary orbit (referred to as the ``Blauuw kick'' \cite{1961BAN....15..265B}). However, compact object natal kicks may amplify the possibility of disruptions or large changes in the binary orbit. With the help of observational constraints on the parameters of a post--SN binary, we can infer its pre--SN parameters. We follow the setup in \citep{2024OJAp....7E..38E}, which in turn uses \citep{1994astro.ph.12023B} for calculating the outcomes of a SN in a binary. These calculations implicitly assume that the mass lost in a SN explosion is instantaneously lost from the system, which is reasonable given the very high ejecta speeds in a SN.

We start by creating an initial population of pre--SN binaries, where we sample over the pre--SN masses and orbital periods. We also assume that the pre--SN binary orbit is circular. This is justified for our scenario, where we model short period binaries that undergo mass transfer, as tides are expected to have circularized the orbit (see \cite{2009MNRAS.400L..20E, 2025ApJ...983...39R,2025arXiv250905243P} for studies on eccentric MT). The post--SN binary properties depend on the pre--SN binary parameters described above, and the details of the SN explosion. The relevant parameters are the amount of mass lost in the SN, and the speed and direction of the compact object natal kick. We also assume that the companion star does not change in mass during the SN \citep{1998A&A...330.1047T,2018PhDT.......211H,2021MNRAS.505.2485O}. This specifies all the relevant parameters needed to generate the post--SN population from the pre--SN binary population.

\subsection{Results} \label{sec:monte-carlo-results}

We use the monte carlo simulations described in Section \ref{sec:monte-carlo-method} to infer the pre--SN properties of the runaway HMXB 4U 1700-37/ HD 153919. We utilize several observational constraints such as the masses of the companion star and the compact object, the current orbital period, and its systemic velocity with respect to its parent cluster NGC 6231. The mass of the progenitor star that forms the compact object is denoted by $m_{1,\mathrm{pre-SN}}$, while its companion has a mass $m_2$. We denote the pre--SN orbital period as $P_{\mathrm{pre-SN}}$. We generate a population of $10^7$ binaries with $m_2$ sampled uniformly between $40-60\,\Mdot$, which roughly spans the estimated mass of the visible companion today. We fix the post--SN mass of the compact object ($m_{1,\mathrm{post-SN}}$) to be $2 \, \Mdot$, and sample the mass lost in the SN explosion ($\Delta m$) uniformly between $0-20 \, \Mdot$. This gives the pre--SN mass of the progenitor of the compact object as $m_{1,\mathrm{pre-SN}} = 2\, \Mdot + \Delta m$. \red{We assume that the mass of the compact object immediately post--SN is similar to its measured mass today, as compact objects in HMXBs are not expected to gain substantial mass through Eddington--limited accretion due to the short timescale of the HMXB phase (suggested citation?).} We sample the pre--SN period of the binary uniformly between $0-30\, \mathrm{days}$. Although we allow for extremely short periods pre--SN, they are not realistic as the stars are unlikely to fit inside a very small orbit. We sample the natal kicks ($v_{\mathrm{kick}}$) to the compact object from a maxwellian distribution with a scale of $265 \, \mathrm{km/s}$ \citep{2005MNRAS.360..974H}. To specify the direction of the kick, we introduce two additional variables $\varphi,\, \theta$. The angle between the kick vector and its projection onto the orbital place is specified by $\theta$, while its orientation with respect to the orbital velocity is denoted by $\varphi$. We assume the kicks to be isotropic in direction. Having defined the pre--SN binary population ($m_{1,\mathrm{pre-SN}}\,, m_2\,, P$) and the relevant parameters for the SN explosion ($\Delta m\,, v_{\mathrm{kick}}\,, \varphi,\, \theta$), we can now compute the properties of the post--SN binary (or disrupted) population. From the post--SN population, we select for systems that are bound ($e<1$) and have gained a systemic velocity ($v_{\mathrm{sys}}$) between $50-80 \, \mathrm{km/s}$ \citep{2021A&A...655A..31V}. We assume that most of the systemic velocity of the XRB with respect to its birth cluster today can be attributed to its post--SN systemic velocity, \red{neglecting the impact of the velocity dispersion of the cluster \citep{2021A&A...645L..10R}. We also implicitly assume that this velocity has not significantly changed during its runaway phase for the past couple of Myr. However, the details of this depend on accurately modeling the effects of the cluster or galactic potential \cite{2025AJ....170..192W}, which we do not model in this work.} The present--day orbital period of the XRB is $ \sim 3.41 \, \mathrm{days}$ \citep{2016MNRAS.461..816I}, but it is possible that the period immediately post--SN period was smaller and has since widened due to wind mass loss from the companion star. To account for this, we select for systems that have a post--SN orbital period between $2-3.41 \, \mathrm{days}$. We choose a lower limit of $2 \, \mathrm{days}$, as shorter periods would imply that the companion star would have filled its Roche Lobe post--SN which does not seem plausible given its current evolutionary state.

We assume that the induced eccentricity in the binary is less than one, meaning that we only select for systems that remain bound post--SN, and do not impose any other constraint on the eccentricity of the binary post--SN.
% (there are disagreements in literature on whether the binary is close to circular or has a modest eccentricity of ~0.2). Since the binary could have had a higher eccentricity post--SN (but less than one) and has undergone tidal circularization since then during the runaway phase, we do not impose any constraints based on its measured eccentricity (controversial) today. [TBD whether to keep this or not]}

\begin{figure}[htbp]
    \centering
    \includegraphics[width=0.47\textwidth]{xrb-monte-carlo-mass-period-kick.pdf}
    \caption{Corner plot showing the distributions of the pre--SN mass, pre--SN period and natal kick strength of the binary population that satisfy the observational constraints of the HMXB 4U1700-37/ HD 153919.}
    \label{fig:xrb_monte_carlo}
\end{figure}

In Fig. \ref{fig:xrb_monte_carlo}, we show a corner plot of the pre-SN mass of the compact object progenitor, the natal kick to its remnant, and the pre-SN orbital period of the systems that satisfy the imposed observational constraints. We emphasize that the probability distribution here is not to be taken literally, as the pre--SN population was not chosen from a physical motivation, and was meant to explore the entire parameter space of interest.

We find that the pre--SN mass of the progenitor of the compact object must have been between $4-13\,\Mdot$ to explain its current orbital period and systemic velocity. Due to the short post--SN orbital period seen today, we also find that the pre--SN orbital period is constrained to be less than $9$ days. Although we do not impose any constraints on the pre--SN orbital period, it is unlikely been very small ($<1-2$ days) as the two stars would not have fit inside the orbit pre-SN. We also find that large pre--SN periods ($>3$ days) require large natal kicks ($>100 \, \mathrm{km/s}$), as it is not possible to shrink binary orbits post--SN period just through mass loss during the SN (Blauuw kick). This is important because MT in binaries is generally expected to lead to orbital widening in most scenarios \citep{2019A&A...624A..66R}. Given that this XRB likely went through a phase of MT in the past (from the progenitor of the compact object onto the current visible star), its pre--SN period was likely large enough that a natal kick is necessarily needed to shrink the orbit post--SN to its current state.

%\red{Another excluded region in this plot is the lower left corner in the pre--SN period vs kick panel. This shows that small enough pre--SN periods and low kicks are incompatible with the imposed observational constraints. The reasons for this are a bit subtle, and we explain it for the case of no kicks, which can then be easily generalized to the presence of kicks. We have imposed a constraint that the post--SN period is larger than $2$ days, and the systemic velocity is smaller than $80\rm km/s$. For a small enough pre--SN period, it so happens that to widen the orbit to satisfy the imposed constraints, the resultant systemic velocity exceeds the observational constraints. However, kicks allow some room for scatter in such situations, and thus the minimum pre--SN period decreases as we increase the strength of the kicks, which creates the diagonal boundary in the lower left corner in the pre--SN period vs kick panel.
%This creates a very narrow range of pre--SN periods ($1.5-2.5$ days), where the observational constraints can be satisfied with small kicks ($v_{\mathrm{kick}} \lesssim 20 \, \mathrm{km/s}$). Small pre--SN periods after an initial phase of stable MT are already quite unlikely given the theoretical expectations of widening of the orbit during MT. This strengthens the case for the requirement for a natal kick at the birth of the $2.5 \,\Mdot$ compact object to have shrunk the orbit, irrespective of whether it is a NS or a low mass BH.}

\section{% Evolutionary %% to fit in one line
\mr{History of 4U 1700-37/ HD 153919}} \label{xrb}

\subsection{Methods} \label{sec:mesa_xrb-methods}

We utilize the pre-SN constraints that we obtained from our monte carlo simulations to reconstruct the past evolutionary history of the HMXB 4U-1700-37/ HD 153919, with detailed binary evolution models, and describe our setup below.

We simulate the evolution of massive binaries using the \mr{stellar
evolution code} \textsc{MESA}
\citep[][\texttt{r24.08.01}]{2011ApJS..192....3P,2013ApJS..208....4P,2015ApJS..220...15P,2018ApJS..234...34P,2019ApJS..243...10P,2023ApJS..265...15J}. Our inlists and choice of input parameters are public (\red{link zenodo/github}) and we \mr{summarize\sout{mention some of the}} relevant parameters and assumptions below.

We use the Ledoux criterion \citep{1947ApJ...105..305L} to find
convective zones, and assume a mixing length parameter
$\alpha_{\mathrm{MLT}}=2.0$. We also include semiconvection
($\alpha_{\mathrm{sc}}=1$) \citep{1983A&A...126..207L}, and
thermohaline mixing ($\alpha_{\mathrm{th}}=1$)
\citep{1980A&A....91..175K}. To account for convective boundary overshooting above the core, we use the exponential model with free parameters $(f,f_0) = (4.25 \times 10^{-2}, 10^{-3})$ \citep{2000A&A...360..952H,2018ApJ...859..100C}, which broadly reproduces the width of the Main Sequence (MS) for a single $16 \, \Mdot$ star \citep{2011A&A...530A.115B}. We do not account for convective boundary over/undershooting for off-center convective layers. To aid in regions at the Eddington limit where convection is inefficient, we utilize the local implicit enhancement of the convective flux in superadiabatic regions \citep[\texttt{use\_superad\_reduction} from][]{2023ApJS..265...15J}.

To limit the computational difficulties in modeling massive
interacting binaries, we include rotation only in the accretor, while
we fix the donor to be non--rotating. The accreted material carries
angular momentum which can spin--up the star, and this is observed in
accretor stars, \mr{for example}, a fraction of runaways
\citep{1993ASPC...35..207B} such as $\zeta\, \mathrm{Ophiuchi}$
\citep{2018AN....339...46Z, 2021ApJ...923..277R}, in Be stars
\citep{2026enap....2..430R} and blue lurkers or stragglers
\citep{2019ApJ...881...47L}. We treat rotation in the ``shellular''
approximation \citep{1992A&A...265..115Z,2012A&A...537A.146E}, where
the angular velocities $\omega$ are constant on isobaric surfaces,
effectively assuming strong angular momentum transport in the
horizontal direction due to turbulence. We start our accretor models
with a $\Omega/\Omega_{\mathrm{crit}} = 0$ at Zero--Age Main Sequence
(ZAMS), but tidal effects tend to \red{[tend to or do they? Say what
happens in your models (possibly quoting the timescale for sync)]}
synchronize the star with the orbital period. We include several
rotational mixing processes in a diffusion approximation, such as
Eddington--Sweet circulation \citep{1950MNRAS.110..548S}, secular and
dynamic shear instabilities, and the Goldreich-Schubert-Fricke (GSF)
instability (see \cite{2000ApJ...528..368H} for a review of these
processes). We assume a Spruit--Taylor dynamo
\citep{2002A&A...381..923S} for treating the transport of angular
momentum, and choose the same parameters as in
\citep{2000ApJ...528..368H}. We also include the rotational
enhancement of the wind mass loss \citep{1998A&A...329..551L} to
\mr{prevent super-critical spin-up of the accretor and regulate the
mass-transfer efficiency \cite[see below and also][]{2025arXiv251115347S}}.

For stars with a surface effective temperature $T_\mathrm{{eff}} > 11000\,\mathrm{K}$, we utilize the mass loss prescriptions from \citep{2023A&A...676A.109B}, unless the surface hydrogen mass fraction falls below $0.4$, in which case we use the optically thick Wolf--Rayet (WR) wind mass loss rates from \citep{2000A&A...360..227N}. For cooler stars with a $\mathrm{T_{eff}} < 10000\,\mathrm{K}$, we use the mass--loss rates from \citep{2024A&A...681A..17D}, and interpolate linearly between the hot and cool winds for the temperature ranges $10000-11000\,\mathrm{K}$ \red{Mention about the condition used for massive RSG}.

We treat mass transfer in a binary using an implicit scheme described
in \citep{1990A&A...236..385K}. We assume that mass transfer is fully
conservative until the accretor reaches a maximum
$\Omega/\Omega_{\mathrm{crit}} = 0.9$, after which the mass--transfer
efficiency depends on the rotationally enhanced wind mass loss. We
assume that the specific angular momentum and entropy of the
transferred mass is equal to the corresponding values at the surface
of the accretor, and the chemical composition is set by the
stratification of the donor. The transferred mass that is not accreted
leaves the binary system carrying with it the specific angular
momentum of the orbital motion of the accretor
\citep{1997A&A...327..620S,2017MNRAS.471.4256V}. \red{[are we going to
discuss parameter variations of these? If so, maybe announce in Sec. X
(or Appendix Y) we discuss variations of these parameters]} We also
include tidal effects \citep{1977A&A....57..383Z,
1981A&A....99..126H,2002MNRAS.329..897H}, which can have a significant
impact on the \mr{spins and orbital} evolution of close binaries.

\subsection{Results}\label{sec:mesa-xrb-results}

%Based on the monte carlo results described in Section \ref{sec:monte_carlo}, we know that the pre--SN mass of the progenitor of the compact object was less than $13\,\Mdot$ and the pre--SN period was less than $9\, \mathrm{days}$. We utilize these constraints to reconstruct the past evolutionary history of the HMXB 4U-1700-37/ HD 153919. We compute binary evolution models with \textsc{MESA}, starting from two stars at ZAMS and evolving them all the way till the initially more massive star (donor) completes helium burning. Afterwards, we detach the binary \red{following \citep{2023ApJ...942L..32R}}, and continue evolving the originally less massive star (accretor) till it finishes hydrogen burning in its core (TAMS).

To illustrate the evolution and structure of our binary models,
we focus our presentation on a fiducial model that can potentially
explain the past evolutionary history of the HMXB 4U 1700-37. This
corresponds to a binary of initial masses $40\,\Mdot$ and $28\,\Mdot$
in a tight orbit with a period of $3\,\mathrm{days}$ \red{[these
numbers here come out of the blue, can we briefly mention the
exploratory work that gives these? This is where showing the grid
could make sense]}. In Fig. \ref{fig:xrb_fiducial_hr}, we show the evolution of the donor and accretor on the HR diagram. Due to the short initial period, the donor overflows its roche lobe after $3.26\,\mathrm{Myr}$, during the Main Sequence (MS) itself. This leads to a phase of mass transfer from the donor to the accretor, referred to as Case A \citep{1967ZA.....65..251K}. This consists of an initial thermal timescale mass transfer ($\sim 0.07\, \mathrm{Myr}$), known as fast Case A (highlighted in blue in Fig. \ref{fig:xrb_fiducial_hr}), where the donor loses a majority of the mass lost during RLOF. As the donor regains thermal equilibrium, it slowly grows in size on a nuclear timescale, which leads to a steady period of mass transfer referred to as slow Case A (highlighted in red in Fig. \ref{fig:xrb_fiducial_hr}). This phase ends around $\sim 5\,\mathrm{Myr}$, which roughly corresponds to when the donor has finished its supply of hydrogen in its core. It starts contracting on a thermal timescale, ending RLOF. At this stage, the donor has lost around $\sim 18\,\Mdot$, \red{with $\sim 7\,\Mdot$ lost during the thermal timescale fast Case A phase}. It now has a mass of $\sim 21.9\,\Mdot$. Due to the stripping of its envelope during MT, the surface has has an elevated helium abundance ($Y = 0.67$) and a hot surface with a $T_{\mathrm{eff}} \sim 42500\,\mathrm{K}$. The mass of the helium core at this point is almost $17.2\,\Mdot$.

On the other hand, the accretor (still on the MS) has grown from an initial $28\,\Mdot$ to $\approx 40.9\,\Mdot$. Due to the increase in mass, it is now overluminous with a $\mathrm{log_{10}}(\mathrm{L}/\Ldot) \approx 5.6$ and a surface temperature of $39400\,\mathrm{K}$ \red{(the end of the highlighted red phase in the bottom panel of Fig. \ref{fig:xrb_fiducial_hr})}.

The orbital period of the binary only widened from an initial period of $3\,\mathrm{days}$ to $\sim 3.68\,\mathrm{days}$ post--MT. \red{The donor star, which still has a thin hydrogen envelope of $\sim 4.7\,\Mdot$, rapidly expands on a thermal timescale during hydrogen shelll burning \citep{2017A&A...608A..11G},} and once again overflows its Roche Lobe. As this occurs after the initial Case A mass transfer, this phase is referred to as Case AB (with the B referencing mass transfer occuring after the MS but before helium depletion \cite{1967ZA.....65..251K}). This phase is highlighted in green in Fig. \ref{fig:xrb_fiducial_hr}. The donor loses an additional $\sim 3.5\,\Mdot$, bringing its total mass down to $18.4\,\Mdot$. Tides prevent the accretor from reaching critical rotation, and it is again able to accrete most of the transferred mass, growing to $\sim 43.9\,\Mdot$. This phase ends just after $\approx 10000\, \mathrm{yr}$.  The donor, now even more stripped of hydrogen (surface $X \sim 0.18$), starts contracting towards higher surface temperatures. Around the same time, it ignites helium in its center, and the core helium burning phase lasts for $\approx 0.44\,\mathrm{Myr}$. During this phase, it has a hot helium rich surface, and would resemble a Wolf--Rayet (WR)\footnote{Note that while WR is a spectroscopic class, we do not compute synthetic spectra. In our stellar evolution models, we switch to WR wind mass loss rates based on the surface temperature and helium abundance, as defined in Section \ref{sec:mesa_xrb-methods}.} star with a strong optically thick wind. These mass loss rates are of the order of $10^{-5} - 10^{-6} \,\Mdot/\rm yr$, and the donor star loses an additional $\sim 5.9\,\Mdot$ during the WR phase until helium depletion. At this point, it is just a few thousand years away from core--collapse \citep{1978ApJ...225.1021W}, and we terminate its evolution in \textsc{MESA}. It has a mass of $\sim 12.5\,\Mdot$, with a Carbon--Oxygen (CO) core mass of $\sim 10.6\,\Mdot$ and a central $\rm ^{12}C/^{16}O \sim 0.38$. There is no hydrogen left on the surface, which mostly consists of helium, carbon and some oxygen. Meanwhile, the originally less massive star (representing the current visible star in the XRB) has a mass of $43.4\,\Mdot$. The orbital period of the binary has increased to $\sim 6.00\,\mathrm{days}$, primarily due to the orbital widening caused by wind mass loss \citep{2000A&A...360..227N}. Both the pre--SN mass of the progenitor of the compact object, the mass of the visible star, and the pre--SN orbital period roughly align with the observations of the XRB, and our expectations of its pre--SN properties from our monte carlo simulations in Section \ref{sec:monte-carlo-results}. At the end of the donor star's life, it may successfully explode and leave behind a $2 \, \Mdot$ compact object, and once the accretor grows large enough that it is close to filling its Roche Lobe, the system would resemble the HMXB 4U 1700-37 that we observe today. Due to previous mass accetion, we find that the surface of the companion star would appear to be enriched in helium and CNO--processed nitrogen, with surface abundances of $\approx 0.42$ and $\approx 0.004$ respectively. In Sections \ref{age}, \ref{vrot}, and \ref{caseA}, we discuss certain discrepancies that we find between our fiducial model and the estimated kinematic age of the HMXB, and the surface rotational velocity of the companion star HD 153919. We also provide further justification for why this system likely underwent Case A MT in the past.

\begin{figure}[htbp]
    \centering
    \includegraphics[width=0.47\textwidth]{xrb_fiducial_hr.pdf}
    \caption{\red{TBA}}
    \label{fig:xrb_fiducial_hr}
\end{figure}

\section{Can the HMXB 4U 1700-37 form a GW source in the future?} \label{xrb-future}

Once the companion star in the HMXB 4U 1700-37 fills its Roche Lobe, it will start transferring mass onto the compact object. However, given the highly asymmetric mass ratio of the system ($q \approx 1/20$), this is likely to become unstable, and lead to a Common Envelope (CE) phase.

\red{We assume mass transfer to become unstable if the mass transfer
rate exceeds $0.1 \, \Mdot/\mathrm{yr}$
\citep[e.g.,][]{Fragos23_posydon}. Such systems are expected to
undergo a Common Envelope (CE) phase.} We do not attempt to model the CE phase directly with \textsc{MESA}, and instead utilize the $\alpha \lambda$ formalism \citep{webbink:84,1990ApJ...358..189D} to study its potential outcome. This depends on the binding energy of the envelope of the star filling its Roche Lobe, as well as the energy sources such as the orbital energy that can be used to unbind the envelope. The effective energy of the envelope, $E_{\text{env}}$, depends on the density structure and internal energy of the star, and is given by --

\begin{equation}\label{eq:BE_def}
    E_{\text{env}}(m) = \int_m^M \text{d}m' \left(-\frac{\text{G}m'}{r(m')} + \alpha_{\text{th}}u(m')\right)
\end{equation}

Here, $m$ refers to the mass coordinate corresponding to the
core--envelope boundary, and $M$ is the total mass of the star. The
first term in the integral refers to the gravitational potential
energy of the mass shell \mr{at coordinate $m'$}. In the second term, $u$ refers to the available internal energy%footnote{This includes recombination energy as well, which we find is only significant in the outer cooler layers of the star, and does not contribute significantly in the integral, which is dominated by mass shells near the core--envelope boundary.}
, and $\alpha_{\text{th}}$ is the fraction of internal energy that can be used to unbind the envelope. If such sources are indeed available to eject the envelope, it reduces the gravitational binding energy and we refer to their sum as the effective energy of the envelope.

The core--envelope boundary is not well defined, and is sensitive to overshooting and other mixing processes in single stars \citep{2020cee..book.....I}. In binaries, rejuvenation can also play an important role in determining the boundary location \citep{2023ApJ...942L..32R}. To account for these uncertainties, we use three different definitions for determining its location, following \citep{2023ApJS..264...45F}. We define the core--envelope boundary as the outermost mass coordinate ($m = M_{\text{core}}$) where the hydrogen mass fraction $X$ is less than $0.01,0.1,0.3$ respectively. The effective envelope energy is often re-parameterized %to make it dimensionless
using $\lambda$ \citep{1990ApJ...358..189D}, which is defined as --

\begin{equation}
    E_{\text{env}} = -\frac{GM_*M_{\text{env}}}{\lambda R_{*}}
\end{equation}

where $M_*$ is the mass of the star, $M_{\text{env}} \equiv M_* - M_{\text{core}}$ is the mass of its envelope, and $E_{\text{env}} \equiv E_{\text{env}}(m = M_{\text{core}})$.

We can then quantify the outcome of a CE phase using energy conservation. This is given by

\begin{equation}
    \alpha_{\mathrm{CE}}\,\Delta E_{\mathrm{orb}} = E_{\mathrm{bind}}
    \label{eq:ce_energy_balance}
\end{equation}

\begin{equation}
    \alpha_{\text{CE}}\left(\frac{G\,M_{\mathrm{core}}\,M_{\mathrm{CO}}}{2a_f}
    - \frac{G\,M_*\,M_{\mathrm{CO}}}{2a_i}\right) = \frac{G\,M_*\,M_{\mathrm{env}}}{\lambda\,R_*}.
    \label{eq:ce_energy_terms}
\end{equation}

\mr{general comment here: I think you don't need to re-explain the
common envelope formalism, you just need to define the symbols and
define your success/failure criterion, following text should be
shortened -- it's not a review on CE!}

where $\alpha_{\text{CE}}$ refers to the fraction of orbital energy that can be used to unbind the envelope, $M_{\mathrm{CO}}$ refers to the mass of the compact object, $a_i$ and $a_f$ refer to the orbital separations before and after the CE phase respectively, and $E_{\mathrm{bind}} = |E_{\mathrm{env}}|$ refers to the effective binding energy of the envelope.

We consider a common envelope phase to be successful if the post--CE orbital separation $a_f$ is large enough such that the remnant core of the star that loses its envelope $R_{\text{core}} \equiv R(m = M_{\text{core}})$ fits inside its Roche--Lobe. This can be rephrased as the minimum $\alpha_{\text{CE}}$ needed for the remnant core to just fill its RL post--CE. If the required $\alpha_{\text{CE}}$ is less than 1, we define the CE phase to be successful, while for values greater than 1, we assume that there is a premature merger between the compact object and the companion's core before the envelope can be ejected. We also do not consider any potential response of the remnant core to the CE phase, and assume that $R_{\text{core}}$ does not change. We also ignore other potential energy sources, such as the energy released by accretion onto the compact object, and energy losses via radiation etc.

To evaluate the outcome of the CE phase, we use our treatment of CE
evolution described above. We save the stellar structure of the
companion star at the onset of RLOF, and compute the binding energy of
its envelope at that instant. The star is still on the MS at that
instant and has a central hydrogen mass fraction of $\approx 0.22$. As
a result, we cannot use the $X=0.01$ or $X=0.1$ definitions for
defining the core--envelope boundary. Using the $X=0.3$ definition, we
get the core--envelope boundary to be located at $\approx
28.6\,\Mdot$, which roughly lies at the boundary of the convective MS
core of the star. Using this boundary location, and the most
optimistic assumption that all of the internal energy in the envelope
is available, we get the effective binding energy of the envelope to
be $\approx 2.35\times 10^{50}\,\mathrm{erg}$. This corresponds to a
$\lambda \approx 0.5$. Using the energy balance Eqn.
\ref{eq:ce_energy_balance}, we find that the minimum
$\alpha_{\mathrm{CE}}$ that would be required to eject the common
envelope, and have the remnant core fit inside its roche lobe is
$\approx 18$. Such a large $\alpha_{\mathrm{CE}}$ implies that the CE
phase is likely going to fail to eject the envelope, and lead to a
premature merger between the star and the compact object. Although
such systems are not GW sources, they could appear as [an energetic
transient \mr{see work on LFBOTs, including Aldana's} or a
Thorne-\.Zytkow Object \mr{\citep[e.g,][]{thorne:75, TZO_review}}]

\section{Formation of GW190814} \label{sec:gw190814}

In the previous section, we constructed detailed binary evolution
models \mr{at ``high'' metallicity} to demonstrate the past evolutionary history of the most asymmetric HMXB in the Galaxy. However, we also showed that it is unlikely to lead to the formation of a highly asymmetric GW source, as the system is likely to merge in a future CE phase due to the star's high envelope binding energy. This motivates us to compute another set of detailed binary evolution models, and explore scenarios in which a potentially CE phase can be successful, and lead to the formation of a highly asymmetric GW source such as GW190814.

As the current LVK detectors probe the relatively distant universe,
the detected GW events likely occured in regions of low metallicity.
For this reason, to describe the formation of a GW190814--like binary,
we compute MESA models at a metallicty of $Z = Z_\odot/10 = 0.00146$
\red{{should i justify this with a reference?}\mr{I'd say it's
justified enough}}.

A key difference in stellar evolution at lower metallicity is that the
mass lost through line--driven winds becomes signifcantly weaker
\citep[e.g.,][]{2001A&A...369..574V,2007A&A...473..603M}. In the
previous section \ref{sec:mesa-xrb-results}, we found that the progenitor of the compact object \red{lost a significant amount of mass during its evolution, and particularly during its core--helium burning phase. This was crucial in lowering its pre--SN mass down to what we inferred from our monte carlo simulations, \mr{\sout{and also potentially aid in its explodability}}}.

Since we now compute binary evolution models at a lower metallicity, we also lower the initial masses of our stars. However, we emphasize that the broad evolutionary pathway remains the same. For our fiducial GW190814--like model, we start with a binary of $27\,\Mdot+25\,\Mdot$ in a 3.5 day orbit.

\subsection{Evolution before first--SN}

The top and bottom panels in Fig. \ref{fig:gw_hr_accretor} show the evolutionary tracks of the donor and accretor in the binary respectively, with the colors highlighting different phases of MT, as defined in Section \ref{sec:mesa-xrb-results}.

As stars at lower metallicity are more compact, mass transfer begins slightly later in the evolution. After $5.9 \,\mathrm{Myr}$, the donor star overflows its Roche--Lobe during the MS, initiating Case A mass transfer. This phase lasts for around $0.75 \,\mathrm{Myr}$. Once the donor reaches TAMS, it starts contracting, and detaches itself from its RL. At this stage, the donor star has a mass of $16.9\,\Mdot$, while the accretor has grown to about $34.9\,\Mdot$. Due to the short orbital period, the strong tidal forces prevent the accretor from reaching critical rotation, and mass transfer remains close to fully conservative. The donor soon re--expands once it starts hydrogen shell burning, and starts the second phase of Case AB mass transfer, similar to our fiducial model in Section \ref{sec:mesa-xrb-results}. At the end of this phase, the binary consists of a $14.4\,\Mdot$ donor, and a $37.4\,\Mdot$ accretor in a tight $6.4\,\mathrm{day}$ orbit. Around the same time, the donor contracts and ignites helium in its core, which lasts for $\sim 0.5 \, \rm Myr$. At core--helium depletion, the donor is left with a mass of $12.7\,\Mdot$, with a carbon-oxygen core mass of $10.7\,\Mdot$ (and a central $\rm ^{12}C/^{16}O \sim 0.39$). On the other hand, the accretor, which is now the more massive star in this system has a mass of $37.3\,\Mdot$.

We assume that in some cases, the donor star successfully explodes at
the end of its life, and leaves behind a NS, or a low mass BH due to
fallback or accretion during the SN. This may resemble the
$2.6\,\Mdot$ secondary in GW190814. \mr{Although we stop the evolution
at helium core depletion, we discuss ``explodability'' in Sec.~\ref{}}. The compact object may also
receive a natal kick at birth, \mr{in analogy with the case of 4U 1700-37}. Depending on its strength and direction, a \mr{variety\sout{plethora}} of post--SN outcomes are possible. This is unlike the Galactic HMXB 4U 1700-37, for which we had observational constraints on the post--SN orbit.
\begin{figure}[htbp]
    \centering
    \includegraphics[width=0.47\textwidth]{gw_hr_accretor.pdf}
    \caption{The evolution of the accretor star on the HR diagram
\red{In which binary? Say it in the caption, mention Z, M1, M2 and P
at ZAMS}. The blue curve highlights Case A mass transfer, while the red curve is for Case AB. The diamond, plus, and star markers correspond to the TAMS, core--helium ignition, and core--helium depletion respectively. The inset panel zooms into the late stages of the core--helium burning phase, where it ascends the RSG branch, and developes a convective envelope (shown in purple)}
    \label{fig:gw_hr_accretor}
\end{figure}

\subsection{Evolution after first--SN}

For systems that remain bound post--SN, the binary consists of a
compact object and a companion star, which was originally the
accretor. Depending on the size of the orbit post--SN and the radial
evolution of the companion star, it may (or may not) fill its Roche
Lobe during its evolution\footnote{\mr{Since we detach the binaries at
the end of He core burning of the initially more massive star,
effectively, we neglect the possible impact of tidal forces after this
point}.}. If it does fill its RL, a phase of reverse--RLOF onto the compact object will ensue. The outcome of this phase sensitively depends on the evolutionary state of the companion star and the stability of mass transfer. The highly asymmetric mass-ratio of the system ($q = 2.6/37.3 \approx 1/14$) is similar to the mass--ratio of the Galactic HMXB that we described in Section \ref{sec:monte-carlo-results}. We once again assume that mass transfer is likely going to be unstable likely going to become unstable, and lead to a CE phase.

\subsubsection{Fate of CE phase}

We use the $\alpha \lambda$ formalism, as described in Section
\ref{xrb-future} to determine the outcome of the CE phase. However, as
the post-SN orbital separation can take a range of values, the density
structure of the star, and correspondingly the binding energy of its
envelope at the onset of CE could vary significantly. \mr{This should
have been explained earlier, and has already been used in the previous
footnote, maybe move to methods}\red{To account for the different possible outcomes post--SN, we detach the companion star from the compact object, and follow its evolution as a single star in \textsc{MESA}. This allows us to model the evolution of its density structure and envelope binding energy along with its radial evolution, thereby accounting for different possible post--SN orbital separations.}

In the top panel of Fig. \ref{fig:gw_be}, we show the effective
binding energy of the envelope of the companion star as a function of
its total stellar radius, \mr{which we use as a proxy for the temporal
evolution}. The blue curve includes the contribution of internal
energy to the effective binding energy of the envelope
\mr{($\alpha_{\rm th}=1$ in Eq.~\ref{eq:BE_def})}, while the golden
color does not include the internal energy \mr{($\alpha_{\rm th}=0$)}, and corresponds to the gravitational binding energy of the envelope. The solid lines corresponds to the $X = 0.1$ limit for defining the core--envelope boundary as described in Section \ref{xrb-future}, while the shaded bands encompass the $X = 0.01$ and $X = 0.3$ limits for defining the boundary. For most of its evolution, as the star grows in radius, there is a moderate decrease in its binding energy. However, during its core--helium burning phase, when it has grown to a size of almost $1250\,\Rdot$, it developes a loosely bound convective envelope and starts ascending the Red Super Giant (RSG) branch. This dramatically decreases the effective binding energy of its envelope, which may make it easier to eject during a potential CE phase \citep{2021A&A...645A..54K}.

Using the effective binding energy of the star's envelope, and the
method described in Section \ref{xrb-future}, we calculate the minimum
$\alpha_{\text{CE}}$ that would be required to eject it as a function of its stellar radius. This is shown in the bottom panel of Fig. \ref{fig:gw_be}, where the line colors and shaded bands follow the same convention as the top panel. We can see that for most of its evolution, when $R_* < 1250\,\Rdot$, the minimum $\alpha_{\text{CE}}$ required to successfully eject the envelope is \mr{much} greater than one. These correspond to the cases where we assume that the CE phase would fail, and there would be a premature merger between the compact object and the core of the star.

However, for $R_* > 1250\,\Rdot$, we find that the minimum $\alpha_{\text{CE}}$ required to eject falls below one. In such scenarios, we assume that the CE phase is successful in ejecting the envelope, and would leave behind a tight, and potentially circularized binary consisting of the remnant core of the star that lost its envelope, and the compact object.

\begin{figure}[htbp]
    \centering
    \includegraphics[width=0.47\textwidth]{gw_fiducial_be_radius.pdf}
    \caption{The top panel shows the evolution of the effective binding energy of the envelope as a function of the total radius of the star. The blue curves include the contribution of internal energy, while the golden curves only include the gravitational binding energy. The solid lines correspond to the $X = 0.1$ definition for the core--envelope boundary, while the shaded bands encompass the $X = 0.01$ and $X = 0.3$ limits. The vertical dash--dot lines denote the radius of the star when it reaches TAMS, when it ignites helium in its core (He ZAMS), and when it finishes core--helium burning (He TAMS). The bottom panel shows the minimum $\alpha_{\text{CE}}$ required to eject the envelope as a function of the stellar radius, with the same color and shading conventions as the top panel.}
    \label{fig:gw_be}
\end{figure}

\subsubsection{Forming a highly asymmetric mass-ratio double compact system}

In our \textsc{MESA} models, we find that for the systems that survive the CE phase, their remnant cores just have $\approx 0.03\,\mathrm{Myr}$ left to the end of core--helium burning, and have a helium core mass of $\approx 20\,\Mdot$. The size of the helium core is $\approx 0.8\,\Rdot$. For an $\alpha_{\mathrm{CE}} = 1$, we find that the post--CE orbital separation is $\approx 2-5\,\Rdot$, depending on whether we  exclude or include the contribution of internal energy. In either case, the remnant core comfortably fits inside the new RL. We assume that the mass and size of the core will not change significantly due to the short time left. At core--collapse, such a heavy core is potentially too large to explode. It will likely implode on itself, and leave behind a $\approx 20\,\Mdot$ BH. These systems now consists of a NS or a low--mass BH with a massive $\approx 20\,\Mdot$ BH companion in a tight orbit of a few solar radii. The subsequent evolution of this system will be dominated by a GW--driven inspiral. For post--CE orbital separations of $2-5\,\Rdot$, we find that it takes at most $\approx 80\,\mathrm{Myr}$ for the inspiral to end in a finale as a compact binary coalescence that may be observed by the LVK detectors \citep{Peters:1963ux}. These signals would consist of a low mass compact object that may be a NS or BH, with a massive BH companion, that would appear similar to the highly asymmetric mass ratio event GW190814.
\mr{Discuss your predictions: since this is a post-CE system, you
expect no eccentricity at formation and thus at merger, since the
pre-CE $q$ was so extreme, we don't expect the BH, formed second, to
spin. These sound trivial to you, but they won't be to the readers and
should be said explicitly!}

 \section{Rates of GW190814--like events} \label{rates}

 In Section \ref{sec:mesa-xrb-results}, we demonstrated that the highly asymmetric mass-ratio Galactic HMXB 4U 1700-37 can form through isolated binary evolution. In Section \ref{sec:gw190814}, we showed that a similar formation pathway works at lower metallicity as well, and can lead to the formation of a highly asymmetric GW source like GW190814 if the CE phase is successful. Assuming an evolutionary link between the two does exist, we can utilize the number of highly asymmetric Galactic HMXBs observed to estimate a back--of--the--envelope rate of highly asymmetric GW events like GW190814. There exist two highly asymmetric mass-ratio HMXBs in the Galaxy. One of them is the Galactic HMXB 4U 1700-37 that we modeled in Section \ref{sec:mesa-xrb-results}, and the other is the X-ray pulsar GX 301-2, which has a $\approx 36-50\,\Mdot$ B supergiant companion, Wray 977 \citep{1995A&A...300..446K}. Assuming that the number of Milky Way equivalent galaxies (MWEG) in the local universe is $\sim 1.16 \times 10^7 \,\rm{Gpc}^{-3}$ \citep{2008ApJ...675.1459K,2010CQGra..27q3001A}, and that the average lifetime of observing a binary in the HMXB phase is $\approx 0.1\, \rm Myr$, the local rate of observing highly asymmetric HMXBs is given by --

 \begin{align}
    \rm \mathcal{R}_{\mathrm{HMXB}} &\sim \frac{2}{\mathrm{MWEG}} \times \frac{1.16 \times 10^7\, \mathrm{MWEG}}{\mathrm{Gpc}^{-3}} \times \frac{1}{0.1\, \mathrm{Myr}} \\
    &= 232\, \mathrm{Gpc}^{-3} \mathrm{yr}^{-1}
 \end{align}

 However, as we demonstrated in Sections \ref{sec:mesa-xrb-results}
and \ref{sec:gw190814}, not all highly asymmetric mass--ratio HMXBs
will form GW190814--like events. We found that a successful CE phase
can occur if the companion star fills its Roche--Lobe when it is a RSG
with a loosely bound convective envelope. This requires the size of
the Roche Lobe after the first--SN to be $\approx 1000-2000\, \Rdot$.
In Fig. \ref{fig:gw_post_SN_rl}, we show the distribution of possible
RL sizes after the first--SN. \mr{why these numbers? can you explain
referring to your detailed model?}We have assumed that a $12.7\,\Mdot$
star explodes to leave behind a $2.6\,\Mdot$ compact object, and it
has a companion of $\approx 37\,\Mdot$ in a $\approx 7\,\mathrm{day}$
orbit. The only free parameters that will now determine the size of
the post--SN RL are the magnitude and direction of the natal kick. We
assume the natal kicks ($v_{\mathrm{kick}}$) to arise from a
\mr{M}axwellian distribution with a scale of $265 \, \mathrm{km/s}$
\citep{2005MNRAS.360..974H} \mr{[would really be better to redo this
with Disberg at least]}. The yellow shaded band corresponds to post--SN Roche Lobes that are smaller than $200\,\Rdot$, which would only appear as a highly asymmetric mass-ratio HMXB, but not form a GW source. The purple shaded band corresponds to post--SN Roche Lobes that are $\approx 1000-2000\, \Rdot$, which will likely have a successful CE phase and form a highly asymmetric mass-ratio GW source. We find that for every $100$ systems that have a post--SN RL $<\,200\,\Rdot$, only $1$ system has a post--SN RL of $\approx 1000-2000\, \Rdot$. Thus we can estimate the local rate of highly asymmetric GW190814--like events to be

 \begin{equation}
    \rm \mathcal{R}_{\mathrm{GW}} \sim 0.01 \times \mathcal{R}_{\mathrm{HMXB}} = 2.32\, Gpc^{-3} yr^{-1}
 \end{equation}

 This back--of--the--envelope estimate is consistent with the
 $\rm 1-23\, Gpc^{-3} yr^{-1}$ rate of GW190814--like events estimate
 by the LVK collaboration \citep{2020ApJ...896L..44A}, based on a single event (GW190814). %, and is a poisson rate.
 However, a detailed rate calculation requires the use of
 population--synthesis techniques to account for the star formation
 history, metallicity dependence, and other uncertainties in linking
 Galactic HMXBs to GW events % , which we defer to future work
\mr{\cite[e.g.,][]{2020ApJ...899L...1Z, 2025arXiv251116648M}}.

 \begin{figure}[htbp]
    \centering
    \includegraphics[width=0.47\textwidth]{gw-post-SN-rl.pdf}
    \caption{The distribution of post--SN Roche Lobe sizes for our
fiducial model for the formation of a GW190814--like system, where we
have assumed the natal kicks to follow a maxwellian distribution with
a $\sigma = 265\,\mathrm{km/s}$. The yellow shaded band corresponds to
post--SN Roche Lobes that are smaller than $200\,\Rdot$, which appear
as the highly asymmetric Galactic HMXBs, but are unlikely to form a GW
source in the future. The purple shaded band corresponds to post--SN
Roche Lobes that are $\approx 1000-2000\, \Rdot$, which will likely
have a successful CE phase and form a highly asymmetric mass-ratio GW
source like GW190814.\mr{how does this look varying $\sigma$ and/or
for a double maxwellian?}}
    \label{fig:gw_post_SN_rl}
\end{figure}

\section{Discussion} \label{sec:highlight}
\mr{General comment: this is a long discussion, almost 3 pages, we
should aim at compressing}
\subsection{Explodability of first--born compact object}\label{sec:explodability}

In Sections \ref{sec:mesa-xrb-results} and \ref{sec:gw190814}, we
described a potential evolutionary pathway to forming highly
asymmetric Galactic XRBs and GW sources. In both scenarios, there are
two keys steps that aid in forming the asymmetric mass-ratio systems.
The first is conservative mass transfer, which allows the mass ratio
of the binary to invert \mr{and the accretor to become very massive
\sout{significantly}}. Secondly, we also require that the the donor
star successfully explodes at the end of its life, and leaves behind a
low mass compact object which could either be a NS or a low mass BH.
\mr{Explain that this is \emph{not} a big hypothesis: the kinematics
of 4U... show there was a kick based on your MC experiment}

However, the explodability of massive stars, and its dependence on its
pre--SN structure, and its past evolutionary history, such as mass
lost through winds or binary interactions is still an active area of
research. While various 3D simulations of core--collapse SNe are
converging on the imporance of the neutrino--driven \mr{convection} in
 driving successful explosions \citep{2025arXiv250214836J}, they are
computationally intensive and can only be done for a few SN progenitors at a time. Additionally, 3D simulations require a realisitic pre--SN structure, which is often computed with 1D stellar evolution models. At present, there is a lack of large grids of 1D pre--SN models at core--collapse, that span a wide range of single and binary star evoluitonary histories, and accurately compute through the late stages of nucelar burning that require large nuclear networks \citep{2016ApJS..227...22F, renzo2024progenitorsmallreactionnetworks}. This has led to the development of various simplified prescriptions and explodability criterion \citep[e.g.,][]{Fryer_2012,Patton_2020,2025A&A...700A..20M} that can be used in large population--synthesis studies to map stars to their remnant properties, such as their mass of the compact object, which can be tied to the natal kick that they recieve.

In our detailed binary evolution models computed with MESA, we only evolve the stars till the end of helium burning. Therefore, we only have access to the mass and composition of the helium core, the carbon--oxygen core, and its density profile at that time. We find that our models have core masses that lie between $\sim 10-13 \, \Mdot$. The explodability of stars with carbon-oxygen core masses between $\sim 7-15 \, \Mdot$ is uncertain, and potentially stochastic \citep{2025A&A...700A..20M}. If such stars do explode, it would appear as a hydrogen--free Type Ib supernova, with ejecta masses of $\approx 8-10\,\Mdot$. This is on the higher end of estimated ejecta masses of observed stripped envelope SNe \citep{2016MNRAS.457..328L}. However, recent studies \citep{2021A&A...645A...5S,2021A&A...656A..58L,2021ApJ...916L...5V} have found that donor stars in binary system may be easier to explode than their single--star counterparts. On the higher mass end, it is usually expected that helium cores $> 15\, \Mdot$ are potentially too large to lead to a successful explosion \citep{Fryer_2012,Patton_2020,2025A&A...700A..20M} (but see also \cite{2018MNRAS.477L..80K,2023ApJ...957...68B,2024ApJ...964L..16B,2025ApJ...987..164B})\footnote{With rotation and magnetic fields, $\gtrsim 100\,\Mdot$ carbon--oxygen cores can potentially explode as well \citep{2022ApJ...941..100S,2025arXiv250815887G}.}. This demonstrates the need for further work in 1D simulations to compute large grids of pre--SN models with large nuclear neetworks that span a wide range of single and binary evolutionary history, as well as a large grid of 3D simulations to better understand the landscape of explodability and develop more accurate prescriptions that can be used in large--scale population synthesis simulations.

\subsection{Spin of the massive BH}

In addition to the masses of the compact objects and the distance to
the source, the GW signal also encodes \mr{information on} the spin of the compact objects. While it is difficult to measure individual spins, the GW detectors are sensitive to a particular mass --weighted combination of the two spins projected onto the orbital angular momentum, called $\chi_{\mathrm{eff}}$. For highly asymmetric mass--ratio systems, where $m_1$ is much larger than $m_2$, the presence of higher--order modes in the GW signal allows breaking certain degeneracies, and also provides a strong constraint on the spin of the massive BH. For GW190814, the spin of the primary BH is tightly constrained to be $\leq 0.07$ \citep{2020ApJ...896L..44A}. Spins of compact objects are closely tied to the evolutionary history of their progenitors, such as the angular momentum (AM) transport between their cores and envelopes, the mass and angular momentum lost through winds and binary interactions, and also the effects of tides in close binaries. Based on our current understanding of AM transport within stars, it is expected that the first born BHs are born with negligible spin. However, it is possible that the progenitor of the second born BH may get tidally spun up before its death through tidal interactions with its companion.

In our evolutionary scenario for GW190814--like systems, we find that the massive BH is the second born compact object. to first order, if ones assumes efficient tidal locking, it may be expected to have a large spin, which would be in tension with the low spin measured of the massive BH in GW190814. However, detailed numerical studies of tides in WR+BH binaries \citep{2023ApJ...952...53M} find that spin--up through tides may not be as efficient as previously assumed. The strength of tidal forces gets progressively smaller as the mass--ratio of the binary decreases, as the tidal effects scale as $q^2$. Additionally, our models undergo a CE phase towards the end of core--helium burning, with only $\approx 0.03\,\mathrm{Myr}$ left till helium depletion in our fiducial model. Therefore, even if we disregard the damping of tidal forces due to the asymmetric mass-ratio of the system, there is likely not enough time for tides to spin up the core of star prior to the end of its life. For these reasons, we emphasize that the low spin of the massive BH in GW190814 need not imply that it is the first--born compact object in the binary.

\subsection{Comparison to previous work}
\mr{to shorten, this could be an appendix too. Try cutting by removing
which code is used, interested readers can go read the refs. Also,
rather than just summarizing what others have found, bring it back to
your own work: your detailed models and the observations of galactic
XRB are a strong confirmation for channels forming the low-mass
compact object first. You don't need to review the other papers
one-by-one, you can just bulk them: isolated channels come in two
flavors and we use XRB and MESA to confirm one of them, briefly
mention dynamics with Lu's channel (triples) and Arca-Sedda again}
Various studies have aimed at understand the potential formation
pathways for forming GW190814--like systems. In the context of
isolated binary evolution, such studies have focused on the mass of
the secondary compact object, and/or the mass--ratio of the event. The
majority of these studies have explored these questions using rapid
binary population synthesis techniques. In particular,
\cite{2020ApJ...899L...1Z} \mr{\sout{utilized the code
\texttt{COMPAS}, and}} found two potential scenarios for forming
GW190814--like systems, \mr{one where the low-mass compact object
forms first, and one where the large BH forms first}\mr{\sout{called
"Channel A" and "Channel B"}}. \mr{\sout{We note that}} The
evolutionary scenario that we explored \mr{here} is qualitatively
similar to \mr{the former \sout{, where the massive BH is the second
born compact object, and } which requires a large kick finely-tuned in
direction  \sout{the first born compact object requires a large natal kick}} to widen the orbit post--SN. However, they find that the rates of such systems in \texttt{COSMIC} are a couple of orders of magnitude lower than the LVK estimated rate of GW190814--like events (\red{should i try explaining?}). The same formation pathway was also found by \citep{2021MNRAS.500.1380M} \mr{\sout{using the code \texttt{COMPAS}}}, but required hertzprung--gap (HG) donors to survive the CE phase. This may not be possible, given the high envelope binding energies while crossing the HG. Additionally, as discussed in Section \ref{sec:gw190814}, we find that donors should be ascending the RSG branch and in the cHeB phase for a successful CE phase. A different evolutionary scenario was proposed by \citep{2025arXiv251116648M}, where they used \texttt{COMPAS} and reverse population synthesis with \texttt{BackPop} \citep{2023ApJ...950..181W} to map GW190814 to an initial binary at ZAMS. Their models prefer starting from an asymmetric binary with masses $\sim 80\,\Mdot+22\,\Mdot$ that undergo CEE in the first phase of MT and form the massive BH first, while the second born compactobject receives a large natal kick, inducing eccentricity in the binary that allows it to merge within a Hubble time (\red{should I be critiquing it?}).

\subsection{Kinematic age of the HMXB 4U 1700-37} \label{age}

Using kinematic and distance measurements with Gaia,
\citep{2021A&A...655A..31V} estimated that the HMXB 4U 1700-37 has a
kinematic age of $\approx 2.2\,\mathrm{Myr}$. This is another observational constraint that can be used to constrain the evolutionary history of this system. While our models for the companion star HD 153919 reasonably match its estimated mass and surface properties, such as its effective temperature and luminosity, we find that a discrepancy in the time taken for it to evolve to its current state post--SN. We find that our models for the companion star only have  $\approx 1\,\mathrm{Myr}$ left on the MS after the death of the progenitor of the compact object, and only $\approx 0.7\,\mathrm{Myr}$ to reach the observed position of HD 153919 on the HR diagram. This implies that there is a factor of $2-3\mathrm{x}$ discrepancy in the age of the system post--SN. Although our models are unable to resolve it, there could be a few potential explanations for addressing this discrepancy.

\begin{enumerate}
    \item The measured systemic velocity of the system is lower than what it was in the past. This could happen if the system has slowed down over time as it escaped the cluster potential well. A higher systemic velocity in the past would imply a smaller kinematic age.
    \item A larger initial donor mass (currently assumed to be $40\,\Mdot$ in our models) would imply a shorter lifetime for the progenitor of the compact object, thereby extending the time spent by the companion star in the post--SN phase. However, a larger donor mass would also imply a slightly larger pre--SN mass of the progenitor of the compact object. This may make it harder to explode and leave behind a $\approx 2\,\Mdot$ compact object. However, as discussed in Section \ref{sec:explodability}, the explodability of stars in this mass range is quite uncertain.
    \item A smaller initial accretor mass (currently assumed to be $28\,\Mdot$) would imply a smaller mass for the star post--SN, thereby extending its MS lifetime. However, this could be in tension with the estimated luminosity and mass of the companion star, as we find that (initially) lower mass accretors are less luminous and less massive at present day than what is inferred from observations.
    \item As the accretor gains mass from the donor, its core may get rejuvenated, and an extra supply of hydrogen in its core can extend its MS lifetime. If rejuvenation is stronger than what we find in our evolutionary models, it may extend the companion star's MS lifetime, and address the discrepancy with the estimated kinematic age of the system.
    % \item A higher systemic velocity than the measured $63\pm 5\, \mathrm{km/s}$ would also allow for a larger pre--SN mass of the donor, as they directly correlate with each other. A larger pre--SN donor mass allows for a larger initial donor mass, which has a shorter lifetime and increases the time spent by the accretor in the post--RLOF phase.
\end{enumerate}

It may also be that the discrepancy could be addressed by a combination of the above factors. Previous studies have also seen a 2$\times$ discrepancy in the ages as measured from single star evolutionary grids and kinematic ages \citep[e.g.,][]{2018A&A...619A..78L}, and further study is warranted to address this issue.

\subsection{Rotational Velocity of HD 153919} \label{vrot}

HD 153919 is the visible companion star to the X--ray source 4U 1700-37. As it is likely a product of past mass transfer, it may have a high surface rotational velocity. However, \citep{2020A&A...634A..49H} used spectral observations and modeling to estimate an upper limit on its projected rotational velocity of $110^{+30}_{-50}\,\mathrm{km/s}$. This implies that HD 153919 is currently sub--synchronous with the orbit. \mr{\sout{The benefit of computing detailed binary evolution models that include rotation is that we can track the evolution of the surface rotational velocity of the companion star throughout its evolution.}} In our models, we find that tidal forces are strong enough to keep the companion star nearly synced to the orbital period, and as a result has a large surface rotation velocity of $\approx 300-400\,\mathrm{km/s}$. However, if we turn off tides in our evolutionary models, we find a much better match with the observed surface rotational velocity of the companion star. In the first phase of mass transfer prior to SN, we found that tidal effects are crucial to keep the accretor star from reaching critical rotation, allowing for nearly conservative MT. The exact strength of tidal forces is currently uncertain, and depends sensitively on the internal structure of the star. The surface rotational velocity of HD 153919 depends on a complex interplay of AM transport during MT, the effects of time--varying tidal effects, and angular momentum lost through winds that are not captured in full detail by our models, and warrants further study, along with better observational constraints.

\subsection{Does the HMXB 4U 1700-37 need to form through Case A mass transfer?} \label{caseA}

Given the short period of the HMXB 4U 1700-37 today, it may have been possible that the progenitor binary underwent a CE phase in the past that shrinked the orbit. However, the companion star in the system (and the originally less massive one) has an estimated mass of $\approx 40-50\,\Mdot$. Since this star likely did not gain appreciable mass during a short CE phase, the original donor star, which was the progenitor of the compact object must have been more massive than $\approx 40-50\,\Mdot$, as unstable mass transfer is thought to occur majorly in systems with an asymmetric mass--ratio. For a CE phase to be successful, the donor star must have been in the post--MS phase. As a result, it would already have a well developed core appropriate for the initial mass of the star, which would potentially be too massive to successfully explode and leave behind a $\approx 2\,\Mdot$ compact object as is seen in the system today. Additionally, based on the results of our monte carlo simulations described in Section \ref{sec:monte-carlo-results}, we know that the pre-SN mass of the progenitor of the compact object was less than $15\,\Mdot$, which is far too small of a core for a star that was much more massive than $\approx 40-50\,\Mdot$. For these reasons, a CE phase in the past is unlikely to explain the current observed properties of the HMXB 4U 1700-37.

Another possibility is that the first phase of MT remained stable, but
occured after the MS of the donor, i.e Case B. \mr{Explain that this
requires a larger initial period} However, this is unlikely to explain the short period of the system as seen today. Based on our monte carlo simulations, we find that the orbital period pre--SN must have been $<10\,\mathrm{days}$, as that is at the limit of how much a strong and directed natal kick can shrink the orbital period. For the mass range relevant to the system of study, Case B occurs for initial periods that are much longer than $10\,\mathrm{days}$. Additionally, \cite{2019A&A...624A..66R} showed that for Case B mass transfer, the orbital period of the binary post--MT is longer than its initial period in most cases, except for scenarios where mass transfer is non conservative ($\beta < 0.6$) \textit{and} there are extreme losses in angular momentum ($\gamma_{\rm RLOF} \geq 2$). Therefore, we find that it is exceedingly difficult for a binary post--Case B MT to shrink significantly post--SN and explain the system's short orbital period today. Additionally, even for Case B, the donor star's core would have already developed before MT occurs, and would be potentially too massive to successfully explode and form a $\approx 2\,\Mdot$ compact object.

\section{Conclusion} \label{sec:conclusion}

To date, GW190814 remains the most asymmetric mass--ratio compact binary merger observed by the LVK detectors. Although there have been several proposed formation pathways, its origin remains unclear. It has also been challenging to model this event as part of population analyses, especially for methods that distinguish between Neutron Stars and Black Holes due to the uncertain nature of the $2.6 \, \Mdot$ secondary compact object.

In this work, we demonstrated the importance of characterizing and understanding the formation of local analogues in trying to understand the origin of GW events, particularly the outliers among them. These systems, which are in various intermediate stages of binary evolution, offer a weatlh of observational constraints that can be utilized to understand their evolutionary history, and are inaccessible when studying solely distant GW sources. We applies this to the highly asymmetric mass--ratio Galactic HMXB 4U 1700-37, and utilized several observables such as its orbital properties, kinematics, and local environment to reconstruct its evolutionary history with the help of monte carlo simulations and detailed binary evolution models. We found that conservative mass transfer on the MS is crucial to invert the mass--ratio of the binary, and explain its current orbit. We later showed that this Galactic HMXB  will likely not lead to the formation of a potential GW source in the near future, and the binary will likely prematurely merge in a potential CE phase. Although not a GW source, this could lead to other exotic transients or even a Thorne--\.Zytkow object.

However, we showed that a similar evolutionary pathway also works at a lower metallicity, and conservative mass transfer on the MS helps in significantly inverting the mass--ratio. In such situations, the lower mass compact object, which could be a NS or a low mass BH forms first. We find that the subsequent CE phase can be successful if the companion star fill its RL when it is on the RSG branch with a loosely bound convective envelope. This requires the orbital separation post--SN to be relatively large, which implies that the first born compact object, regardless of whether it is a NS or a BH, to receive a strong natal kick. The CE phase significantly hardens the binary, and the remnant core of the companion star likely implodes to form a massive BH. The subsequent evolution is dominated by a GW--driven inspiral, and end in a compact binary coalescense that would appear as a highly asymmetric GW source to the LVK detectors.

As the LVK detectors rapidly increase their sample of detected events future observing runs, it would enable a better characterization of highly asymmetric mass-ratio BBH or NSBH systems. It would allow us to explore correlations in their spin, mass, and redshift distributions, which could potentially distinguish between the different formation pathways proposed for such systems. Additionally, several ongoing or future EM surveys \red{such as,,,,} will rapidly increase the sample of massive binaries in various intermediate stages of evolution. These systems, which can be probed with a wealth of observational techniques, provide a local anchor to understand various uncertainties in the evolution of massive stars and compact objects in binaries, and help us better understand the origin of their distance universe GW counterparts.

\begin{acknowledgments}

\end{acknowledgments}

\section{Software}

This work made use of the following software packages: \texttt{astropy} \citep{astropy:2013, astropy:2018, astropy:2022}, \texttt{Jupyter} \citep{2007CSE.....9c..21P, kluyver2016jupyter}, \texttt{matplotlib} \citep{Hunter:2007}, \texttt{numpy} \citep{numpy}, \texttt{pandas} \citep{mckinney-proc-scipy-2010, pandas_17229934}, \texttt{python} \citep{python}, \texttt{scipy} \citep{2020SciPy-NMeth, scipy_17101542}, \texttt{Bilby} \citep{bilby_paper, bilby_paper_2, Bilby_17314023}, and \texttt{corner.py} \citep{corner-Foreman-Mackey-2016, corner.py_14209694}.

This research has made use of the Astrophysics Data System, funded by NASA under Cooperative Agreement 80NSSC21M00561.

This work uses Modules for Experiments in Stellar Astrophysics \citep[MESA][]{Paxton2011, Paxton2013, Paxton2015, Paxton2018, Paxton2019, Jermyn2023}. The MESA EOS is a blend of the OPAL \citep{Rogers2002}, SCVH \citep{Saumon1995}, FreeEOS \citep{Irwin2004}, HELM \citep{Timmes2000}, PC \citep{Potekhin2010}, and Skye \citep{Jermyn2021} EOSes. Radiative opacities are primarily from OPAL \citep{Iglesias1993, Iglesias1996}, with low-temperature data from \citet{Ferguson2005} and the high-temperature, Compton-scattering dominated regime by \citet{Poutanen2017}. Electron conduction opacities are from \citet{Cassisi2007} and \citet{Blouin2020}. Nuclear reaction rates are from JINA REACLIB \citep{Cyburt2010}, NACRE \citep{Angulo1999} and additional tabulated weak reaction rates \citet{Fuller1985, Oda1994, Langanke2000}. Screening is included via the prescription of \citet{Chugunov2007}. Thermal neutrino loss rates are from \citet{Itoh1996}. Roche lobe radii in binary systems are computed using the fit of \citet{Eggleton1983}. Mass transfer rates in Roche lobe overflowing binary systems are determined following the prescription of \citet{Ritter1988}.\footnote{Note this is only valid for the default settings of MESA, adjust as necessary for custom settings used.}

Software citation information aggregated using \texttt{\href{https://www.tomwagg.com/software-citation-station/}{The Software Citation Station}} \citep{software-citation-station-paper, software-citation-station-zenodo}.

\appendix

\section{Resolution Tests}

To ensure that our results are robust to the choice of spatial and
temporal resolution \mr{\sout{of mesh points} in \textsc{MESA}}, we
repeated our fiducial binary evolution calculations by increasing the number of mesh points \mr{by} decreasing \texttt{mesh\_delta\_coeff} and \texttt{mesh\_time\_coeff} by a factor of $2/3$. In Fig \ref{fig:resolution_test}, we show that the evolution of the donor and accretor stars on the HR diagram for our fiducial models to explain the evolutionary history of the HMXB 4U 1700-37 (left panel) and a GW190814--like system (right panel). The blue and red solid colors represent the default resolution for the donor and accretor, where we have $\mathrm{\texttt{mesh\_delta\_coeff}} \equiv \Delta_x = 1$ and $\mathrm{\texttt{mesh\_time\_coeff}} \equiv \Delta_t = 0.8$. The cyan and pink dashed lines represent higher resolution models, where we decrease both $\Delta_x$ and $\Delta_t$ by a factor of $2/3$. We find that the evolutionary tracks of both the donor and accretor stars are nearly identical for both choices of resolution, demonstrating that our results are robust and there is convergence in the numerical treatment of binary evolution in \textsc{MESA}.
\mr{Mention number of mesh points in the final models and number of
timesteps at each resolution in the text}.

\begin{figure}
\gridline{\fig{xrb_hr_resolution_test.pdf}{0.47\textwidth}{(a)}
\fig{gw_hr_resolution_test.pdf}{0.47\textwidth}{(b)}}
\caption{(a) The evolution of the accretor and donor star, modeling the evolutionary history of the HMXB 4U 1700-37 on the HR diagram for two different choices of resolution. (b) The evolution of the accretor and donor star, modeling the formation of GW190814 on the HR diagram for two different choices of resolution.} \label{fig:resolution_test}
\end{figure}

\bibliography{xrb_gw}{}
\bibliographystyle{aasjournalv7}

\end{document}
%%% Local Variables:
%%% mode: LaTeX
%%% TeX-master: t
%%% End:
